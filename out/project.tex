% !TeX spellcheck = en-US
% !TeX encoding = utf8
% !TeX program = pdflatex
% !BIB program = biber


% vv  scroll down to line 200 for content  vv


\let\ifdeutsch\iffalse
\let\ifenglisch\iftrue
\input{pre-documentclass}
\documentclass[
    % fontsize=11pt is the standard
    paper=a4,
    oneside,  % we are optimizing for both screen and two-side printing. So the page numbers will jump, but the content is configured to stay in the middle (by using the geometry package)
    bibliography=totoc,
    %               idxtotoc,   %Index ins Inhaltsverzeichnis
    %               liststotoc, %List of X ins Inhaltsverzeichnis, mit liststotocnumbered werden die Abbildungsverzeichnisse nummeriert
    headsepline,
    cleardoublepage=empty,
    parskip=half,
    %               draft    % um zu sehen, wo noch nachgebessert werden muss - wichtig, da Bindungskorrektur mit drin
    draft=false
]{scrbook}
\input{config}


\usepackage[
    title={Using Games to collect Data for Reinforcment Learning },
    author={Markus Baur\\
2872854},
    type=Forschungsarbeit,
    institute={Institut für Computerphysik},
    course={Simulation Technology},
    examiner={Prof. Dr. Christian Holm},
    supervisor={Dr. rer. nat. Samuel James Tovey},
    startdate={Jan 15, 2025},
    enddate={Jun 13, 2025}
]{scientific-thesis-cover}

% Hier stehen alle Abkürzungen

\newacronym{rl}{RL}{reinforcement learning}
\newacronym{marl}{MARL}{multi-agent reinforcement learning}
\newacronym{ppo}{PPO}{proximal policy optimization}
\newacronym{tanh}{tanh}{tangens hyperbolicus}
\newacronym{relu}{relu}{rectified linear unit}


\makeindex

\begin{document}

%tex4ht-Konvertierung verschönern
\iftex4ht
    % tell tex4ht to create picures also for formulas starting with '$'
    % WARNING: a tex4ht run now takes forever!
    \Configure{$}{\PicMath}{\EndPicMath}{}
    %$ % <- syntax highlighting fix for emacs
    \Css{body {text-align:justify;}}

    %conversion of .pdf to .png
    \Configure{graphics*}
    {pdf}
    {\Needs{"convert \csname Gin@base\endcsname.pdf
            \csname Gin@base\endcsname.png"}%
        \Picture[pict]{\csname Gin@base\endcsname.png}%
    }
\fi

%Tipp von http://goemonx.blogspot.de/2012/01/pdflatex-ligaturen-und-copynpaste.html
%siehe auch http://tex.stackexchange.com/questions/4397/make-ligatures-in-linux-libertine-copyable-and-searchable
%
%ONLY WORKS ON MiKTeX
%On other systems, download glyphtounicode.tex from http://pdftex.sarovar.org/misc/
%
\input glyphtounicode.tex
\pdfgentounicode=1

%\VerbatimFootnotes %verbatim text in Fußnoten erlauben. Geht normalerweise nicht.

\input{commands}
\pagenumbering{arabic}
\Titelblatt

%Eigener Seitenstil fuer die Kurzfassung und das Inhaltsverzeichnis
\deftripstyle{preamble}{}{}{}{}{}{\pagemark}
%Doku zu deftripstyle: scrguide.pdf


% BEGIN: Verzeichnisse

\iftex4ht
\else
    \microtypesetup{protrusion=false}
\fi

%%%
% Literaturverzeichnis ins TOC mit aufnehmen, aber nur wenn nichts anderes mehr hilft!
% \addcontentsline{toc}{chapter}{Literaturverzeichnis}
%
% oder zB
%\addcontentsline{toc}{section}{Abkürzungsverzeichnis}
%
%%%

%Produce table of contents
%
%In case you have trouble with headings reaching into the page numbers, enable the following three lines.
%Hint by http://golatex.de/inhaltsverzeichnis-schreibt-ueber-rand-t3106.html
%
%\makeatletter
%\renewcommand{\@pnumwidth}{2em}
%\makeatother
%
\tableofcontents

% Bei einem ungünstigen Seitenumbruch im Inhaltsverzeichnis, kann dieser mit
% \addtocontents{toc}{\protect\newpage}
% an der passenden Stelle im Fließtext erzwungen werden.

%\listoffigures
%\listoftables

%Wird nur bei Verwendung von der lstlisting-Umgebung mit dem "caption"-Parameter benoetigt
%\lstlistoflistings
%ansonsten:

%\listof{Listing}{List of Listings}

%mittels \newfloat wurde die Algorithmus-Gleitumgebung definiert.
%Mit folgendem Befehl werden alle floats dieses Typs ausgegebe\listof{Algorithmus}{List of Algorithms}

%\listofalgorithms %Ist nur für Algorithmen, die mittels \begin{algorithm} umschlossen werden, nötig

% Abkürzungsverzeichnis
%\setglossarystyle{long}
%\printnoidxglossaries

\iftex4ht
\else
    %Optischen Randausgleich und Grauwertkorrektur wieder aktivieren
    \microtypesetup{protrusion=true}
\fi

% END: Verzeichnisse


% Headline and footline
\renewcommand*{\chapterpagestyle}{scrplain}
\pagestyle{scrheadings}
\pagestyle{scrheadings}
\ihead[]{}
\chead[]{}
\ohead[]{\headmark}
\cfoot[]{}
\ofoot[\usekomafont{pagenumber}\thepage]{\usekomafont{pagenumber}\thepage}
\ifoot[]{}

\input{tightlist.tex}

\DefineVerbatimEnvironment{Highlighting}{Verbatim}{commandchars=\\\{\}}

\include{pandoc} % pandoc syntax highlithing



%% vv  scroll down for content  vv %%























%%%%%%%%%%%%%%%%%%%%%%%%%%%%%%%%%%%%%%%%%%%%%%%%%%%%%%%%%%%%%%%%%%%%%%%%%%%%%%
%
% Main content starts here
%
%%%%%%%%%%%%%%%%%%%%%%%%%%%%%%%%%%%%%%%%%%%%%%%%%%%%%%%%%%%%%%%%%%%%%%%%%%%%%%


\chapter{Introduction}
\hypertarget{introduction}{%
\section{Introduction}\label{introduction}}

\begin{itemize}
%\tightlist
\item
  OpenMP
\item
  Task Parallel execution
\item
  simple way to speed up existing OpenMP programms

  \begin{itemize}
%  \tightlist
  \item
    no complete rewrite with MPI/etc
  \end{itemize}
\end{itemize}


\chapter{Theory}
\input{chapters/this_work}

\chapter{Game Implementation}
\input{chapters/game}

\chapter{Machine Learning Implementation}
\input{chapters/learning}

\chapter{Results}
\input{chapters/results}

\chapter{Conclusion}
\input{chapters/conclusion}
\clearpage

%\printindex

\printbibliography

All links were last followed on \today.

\cleardoublepage

\chapter{Appendix}
\input{chapters/appendix}

\pagestyle{empty}
\renewcommand*{\chapterpagestyle}{empty}
\Versicherung
\end{document}
